\section{Data as Matrices}\label{pre:dam}

Matrices are ubiquitous in data\hyp{}related fields, from the 
long\hyp{}established computer science and statistics, to the rapidly growing
machine learning and data science. From different perspectives, a matrix can
represent a variety of objects:
\begin{itemize}
	\item A simple and compact way of storing information. For example, each
	column of a matrix corresponds to a data point and each row stands for a
	feature.
	\item A linear map $A\Colon\calV\to\calU$ from one vector space to another
	through matrix\hyp{}vector\hyp{}multiplication $v\mapsto u = Av$. 
	\item A bilinear map $A\Colon\calV\times\calU\to \BBR$ such that $(v,
	u)\mapsto u^TAv$.
\end{itemize}
Most importantly, these representations are not incompatible with each other,
and it is the capacity of being all of them at once that makes matrices such
a powerful tool in applications. For instance, an adjacency matrix $A\in\BBR^
{N\Times N}$ can describe an unweighted graph where $A_{ij} = 1$ if node $i$
and $j$ are connected by an edge, and $0$ otherwise. Each row of $A$ can be
viewed as the connectivity features for the corresponding node. Alternatively,
applying $A$ to a vector standing for the initial condition of a diffusion
process captures the dynamics on this graph. Finally, the bilinear map of $A$ on
$\calV = \calU = \{0, 1\}^N$ measures the number of edges from one set of nodes
to the other and thus can be used in graph partitioning tasks.

Therefore, it is inadequate to categorize data matrices by their linear
algebraic roles. Instead, we introduce the matrices encountered in this thesis
by the properties of datasets they encapsulate. Regardless the type of data
matrices we are dealing with, the most critical task is identifying the
intrinsic numerical properties attached to it, so that we can adapt our
algorithms to specific applications.

\paragraph{Matrices for Feature}

A dataset with finite number of features can be associated with a vector space,
i.e., feature space. An object\hyp{}feature matrix $A\in\BBR^{M\Times N}$
compactly encodes all of its information by treating $N$ data points as column
vectors, each with $M$ features. In addition, many algorithms involve linear
algebraic operations on data, and such representation naturally facilitates
their computation as well as massive parallelization. For example, data
compression and dimensionality reduction techniques, such as frequency sampling
and principal component analysis, are usually implemented as low\hyp{}rank
decompositions on data matrices. Neural networks can be viewed as a sequence of
linear transformations and element\hyp{}wise operations, which rely on the
matrix computation for efficient parallelization. As a result, matrices together
with higher order variations are easily the most popular mathematical objects
for keeping track of data during computation.

In this dissertation, a majority of matrices are this type even though they are
not always the focus of study. Every experiment of Gaussian processes regression
uses object\hyp{}feature matrices for both input and output. In topic modeling,
the topic\hyp{}word matrix is a latent matrix of this type, and plays an crucial
role in interpreting the model. Finally, the main matrix of interest in 
\cref{ch6} is formed by having electron orbital pairs as objects and 
real\hyp{}space discretization as features.

\paragraph{Matrices for Correlation}

This type of matrices describes the pairwise relationship of objects rather
than individual ones. Examples in this dissertation include the 
co\hyp{}occurrence matrix in \cref{ch5} that indicates the probability of two
objects occurring together in a document, and the kernel matrices in \cref{ch4}
that measures the covariance between two points for a Gaussian process. The
former can be derived from the document\hyp{}object matrix, which suggests this
type of data matrices can sometimes be considered as higher\hyp{}order
information extracted from the object\hyp{}feature matrices. The correlation
between objects has various benefits and is widely used in statistical models.
In our case, it leads to uncertainty quantification, theoretical optimality
guarantee, and algorithmic robustness.

\paragraph{Matrices for Dynamics}

The last type of data matrices is generally associated with differential
operators, whether in a continuous setting or the corresponding discrete
analogs. These matrices appear in \cref{ch6} both explicitly as discretization
of Laplace operator, and implicitly as integral operators---e.g., the 
Hatree\hyp{}Fock operator. Here, we focus on exploiting structural information
to compress these matrices and gain computational efficiency. On the other hand,
in \cref{ch3} the graph Laplacian and the random walk matrix describe the
dynamics on the original graph, which we use to reveal structural properties.
There is no doubt a deep connection between the dynamics on an underlying object
and its structures. In general, we can leverage our knowledge on one to gain
valuable insights on the other.

\section{Functions of Matrices}\label{pre:fom}

The main mathematical problems in both \cref{ch3,ch4} can be simplified to
estimating key quantities for functions of matrices. In this dissertation, a
function of matrices refers to $f(A)$, produced by a spectral mapping of $f$ on
a matrix $A\in\BBR^{N\Times N}$. \citet{higham2008functions} gives a rigorous
and general definition of $f(A)$, which only requires $f$ to be \emph{defined}
\footnote{Defined is a formal term here. Given a matrix $A$ with distinct
eigenvalues $\lambda_1,\cdots,\lambda_s$ and $n_i$ the size of the largest
Jordan block for $\lambda_i$, the function $f$ is defined on the spectrum of $A$
if $f^{(j)}(\lambda_i)$ exists for $j = 0\Colon n_i-1$ and $i = 1\Colon s$.
Please refer to \cite[Definition~1.1]{higham2008functions} for a detailed
discussion.} on the spectrum of $A$. However, we mostly work with a special
case, where $A$ is Hermitian and $f$ is smooth over the spectrum of $A$. Given
the eigendecomposition $A = V^T\Lambda V$ where $V$ is orthogonal and $\Lambda =
\diag(\lambda_1,\cdots,\lambda_N)$ is diagonal, the spectral mapping theorem
tells us
\begin{equation}\label{eqn:smt}
	f(A) = f(V^T\Lambda V) = V^T f(\Lambda)V = V^T\diag(f(\lambda_1),\cdots,f
	(\lambda_N))V\,.
\end{equation}

Historically, there has been long\hyp{}lasting interest in numerical methods for
computing functions of matrices because of its importance in many applications,
such as differential equations, Markov models, and network science. The
straightforward approach in \cref{eqn:smt} takes an $\calO(N^3)$
eigendecomposition, which is not always feasible. Many tailored algorithms have
been developed for a few specific but prevalent functions, like the matrix
square root and matrix exponential \cite[Chapter 6 \& 10]{higham2008functions}.
In a remarkable work, \citet{moler2003nineteen} presented nineteen ways of
computing matrix exponential, comparing these in terms of both efficiency and
stability. Generally, matrix decomposition methods are considered to be the most
practical with existing efficient implementations.

Given the context in this dissertation, we adopt two methods for evaluating
functions of matrices. The first is a polynomial expansion method in the
Chebyshev polynomial basis. The Chebyshev polynomials $T_m(x)$ can be defined
with the recurrence:
\begin{equation}\label{eqn:cheb_3term}
	T_0(x) = 1,\quad T_1(x) = x,\quad T_{m+1}(x) = 2xT_m(x) - T_{m-1}(x)\,.
\end{equation}
The corresponding expansion is simply $f(x) \approx \tilde{f}(x)= \sum_
{m=0}^Mc_mT_m(x)$. It is well\hyp{}known that such expansion with appropriate
weights $c_m$ produces the best polynomial approximation over the compact
support $[-1,1]$. In particular, for any smooth function $f$, the approximation
error decays exponentially with the number of terms in the expansion~\cite{
Trefethen-2013-ATAP}. Many functions we deal with satisfy the smoothness
condition, thereby making this method very attractive in our applications. To
evaluate the function of matrix $f(A)$, we take advantage of the three\hyp{}term
recurrence relation
\begin{equation}\label{eqn:3term_fom}
	T_{m+1}(A) = 2AT_m(A)-T_{m-1}(A)\,,
\end{equation}
and avoid storing all $T_m(A)$ at once. When we are only interested in applying
$f(A)$ to a vector $v$, we can further reduce the computation to just
calculating $T_m(A)v$ for $m = 1,\cdots, M$, and exploit fast MVM with $A$ if
possible.

The second method is Gauss quadrature and Lanczos (GQL) algorithm proposed by 
\citet{golub1997matrices}. It is based on the Lanczos algorithm, which produces
the decomposition
\begin{equation}\label{eqn:lan_decomp}
	AZ_M = Z_M\Gamma_M + r_Me_M^T\,,
\end{equation}
where $Z_M^TZ_M = I_M$, $Z_M^Tr_M = 0$, and $\Gamma_M$ tridiagonal. If $A$ is
numerically low\hyp{}rank, or has relatively few distinct eigenvalues, it will
only take a small number of Lanczos iterations to obtain the approximation
$A\approx Z_M\Gamma_MZ_M^T$. Afterwards, we can quickly compute $f(\Gamma_M)$
through eigendecomposition since $\Gamma_M$ is $M$\hyp{}by\hyp{}$M$ and
tri\hyp{}diagonal. The final approximation is $f(A) = Z_Mf(\Gamma_M)Z_M^T$.

Both methods we introduce here have clear advantages and drawbacks. The
Chebyshev expansion method has strong theoretical guarantees on convergence rate
for $f$ with some degree of smoothness over the spectrum of $A$. For irregular
$f$, it may take extremely large number of moments to resolve possible
discontinuities, thus suffering in either efficiency or approximation quality.
On the other hand, GQL works well only for matrices with a few distinct
eigenvalues, which allows the Lanczos algorithm to quickly converge. Therefore,
it is crucial for us to determine the numerical characteristics of the matrices
we are interested in, and choose the appropriate method for each application.

\section{Stochastic Estimation}\label{pre:ste}

Stochastic estimation is one of the key tools in this dissertation. Given a
linear operator $A\in\BBR^{N\Times N}$, often in implicit form, it helps us to
efficiently extract the trace and diagonal by probing the operator with random
vectors. 

The stochastic trace estimator was first proposed by~
\citet{hutchinson1990stochastic}, who used the technique to approximate the
trace of the influence matrix for Laplacian smoothing splines. In his case,
applying influence matrix $A$ to each vector requires solving an expensive
smoothing spline; obtaining all diagonal elements exactly --- by calculating
$e_i^TAe_i$ for $i=1,\cdots, N$ and $e_i$ the $i$\hyp{}th column of the identity
matrix --- had a significant cost. Instead, he applied $A$ to a set of
independent probe vectors $z$ such that $z_i$'s are i.i.d. with mean $0$ and
variance $1$. Consequently, 
\begin{equation}\label{eq:probe_trace}
	\BBE[z^TAz] = \sum_{i,j}A_{ij}\BBE[z_iz_j] = \tr(A)\,.
\end{equation}
Choosing $N_z$ independent probe vectors $Z_j$, we obtain the unbiased estimator
\begin{equation}\label{eqn:trace_est}
	\tr(A) = \BBE[z^TAz] \approx \frac{1}{N_z}\sum_{j=1}^{N_z}Z_j^TAZ_j\,.
\end{equation}
Hutchinson also identified sampling $z_i$ from the  Bernoulli distribution leads
to minimal variance on the estimated trace.

Since then, \citet{avron2011randomized} reviewed many possible choices of probes
for \cref{eq:probe_trace,eq:probe_diag}; another common choice is vectors with
independent standard normal entries. On the other hand,  
\citet{bekas2007estimator} has extended the idea for diagonal estimation by
observing
\begin{equation}\label{eq:probe_diag}
	\BBE[z\odot Az] = \diag(A)\,,
\end{equation}
where $\odot$ represents the Hadamard (elementwise) product. Let $Z\in\BBR^
{N\Times N_z}$ so the columns are independent probe vectors. If we inspect the
$k$\hyp{}th diagonal element,
\begin{equation}\label{eqn:exact_cond}
	A_{kk} = \BBE[z_k(Az)_k] = \frac{1}{N_z}\sum_{i}A_{ki}\sum_{j=1}^{N_z}Z_{kj}Z_
	{ij}
\end{equation}
Based on \cref{eqn:exact_cond}, \citet{bekas2007estimator} proposed the exact
condition for the diagonal estimator.

\begin{theorem}[Exact Condition for Diagonal Estimatior]\label{thm:exact_cond}
Let $Z\in\BBR^{N\times N_z}$ be the matrix whose columns are probe vectors of
the diagonal estimator in \cref{eqn:exact_cond}. Use $Z_{i\ast}$ to denote the
$i$\hyp{}th row of $Z$. The estimation of $A_{kk}$ is exact if $\norm{Z_
{k\ast}} = 1$, and $Z_{k\ast}^TZ_{i\ast} = 0$ whenever $H_{ki}\neq 0$ for $i\neq
k$.
\end{theorem}

The direct consequence of \cref{thm:exact_cond} is that we can now design a
specific set of probe vectors to satisfy the exact condition by solving a graph
coloring problem. Treating $A$ as the adjacency matrix of a graph, and the
non\hyp{}zero entries as edges, a graph coloring partitions the nodes into
disjoint subsets for which no edge exists between each other. If we assign each
subset a row vector chosen from an orthonormal set, the resulting $Z$ will yield
the exact diagonal through the same calculation in \cref{eqn:exact_cond}.
Although obtaining an optimal coloring scheme is NP\hyp{}hard, there exists a
greedy coloring algorithm using breadth\hyp{}first\hyp{}search that can produce
reasonable results for some applications \cite[Theorem~28.33]
{arumugam2016handbook}.

Finally, we also attempted to use control variate to improve the stochastic
estimation. In \cref{ch4} where the goal is to estimate the predictive variance,
we take the more accurate results from pivoted Cholesky approximation on a few
data points as control variate. Despite little reduction in variance, we
are able to obtain an unbiased estimator.

\section{Interpretable Low\hyp{}Rank Approximations}\label{pre:lra}

Low\hyp{}rank structure of matrices plays an essential role in many of our
methods. If a matrix $A\in\BBR^{N\times N}$ has rank $k$, it can be
decomposed into a product of two smaller matrices $W\in\BBR^{N\times k}$ and
$H\in\BBR^{k\times N}$:
\begin{equation}\label{eqn:low_rank}
A = WH\,.
\end{equation}
Hence, we are able to efficiently store $A$ and multiply it with a vector in
$\calO(N)$ rather than $\calO(N^2)$. Furthermore, the columns of matrix $W$
provide us useful insight about the data, since they form a basis that span the
original columns of $A$. Low\hyp{}rank approximation is certainly one of the
most well\hyp{}studied problems in linear algebra; and it is becoming increasing
popular with the recent development in data science and machine learning. 
High\hyp{}dimensional data matrices encountered in these fields commonly have a
low numerical rank. Correspondingly, researchers have proposed low\hyp{}rank
models for a wide variety of applications, such as principal component analysis,
natural language processing, and recommender system. \citet{udell2019big}
formalized this notion by showing a nice latent variable model produces data
that can be well approximated by low\hyp{}rank matrices.

Among the extensive literature on computing and exploiting low\hyp{}rank
structures, there is a growing interest in interpretability. Previously, 
low\hyp{}rank approximations have aimed to capture the information with the
highest numerical importance, but often fail to preserve intrinsic properties of
datasets, e.g., non\hyp{}negativity, sparsity, and convexity. Thus, many studies
developed constrained low\hyp{}rank approximation algorithms in order to retain
these properties, and allow us to interpret the output in a natural setting. A
prominent example is the non\hyp{}negative matrix approximation, for which $W$
and $H$ are required to be element\hyp{}wise non\hyp{}negative. 
\citet{lee1999learning} applied it to images of faces, and the produced factors
can be visualized as components of facial structures. In \cref{ch5}, the topic
model is also a non\hyp{}negative low\hyp{}rank decomposition, and we are able
to analyze each topic as a probability distribution over words.

One attractive approach towards interpretable low\hyp{}rank approximations
enforces $W$ to consist of columns from $A$. This allows us to describe all data
points in terms of a small subset, maintaining the fundamental characteristics.
The CUR matrix decomposition, a well\hyp{}known example of this type, is often
preferred over singular valued decomposition in certain applications for its
efficiency and interpretability. The biggest challenge here is to select a good
subset of data points to obtain an accurate representation. The column subset
selection problem is also a heavily discussed topic in numerical linear algebra
\cite{boutsidis2009improved,deshpande2010efficient}. One of the most popular
building block for solving it is QR factorization with column pivoting (QRCP).
The column pivoting procedure gives QR factorization the rank\hyp{}revealing
property. Combined with randomized sampling techniques, QRCP is able to
effectively select columns that leads to near\hyp{}optimal low\hyp{}rank
approximations. Both \cref{ch5,ch6} tackle the column subset selection problem,
but from different perspectives. \cref{ch5} develops a scalable 
pre\hyp{}processing framework to overcome the robustness issue for QRCP on a
noisy dataset. \cref{ch6} exploits the physical attributes in the system to
replace pivots from QRCP with centroids from the weighted K\hyp{}means
algorithm. The resulting decomposition is much more efficient without any loss
in accuracy.