One standard approach to scaling GPs substitutes the exact kernel with an
approximate kernel. When the GP fits values and gradients, one may attempt to
separately approximate the kernel and the kernel derivatives. Unfortunately,
this may lead to indefiniteness, as the resulting approximation is no longer a
valid kernel. Instead, we differentiate the approximate kernel, which preserves
positive definiteness. We do this for the SKI and SKIP kernels below, but our
general approach applies to any approximate MVM.

\subsection{D-SKI}

D\hyp{}SKI (SKI with derivatives) is the standard kernel matrix for GPs with
derivatives, but applied to the SKI kernel.  Equivalently, we differentiate the
interpolation scheme:
\begin{equation}\label{eqn:weightderiv}
  k(x,x') \approx \sum_i w_i(x)k(x_i,x')\rightarrow \nabla k(x,x') \approx
  \sum_i \nabla w_i(x) k(x_i,x')\,.
\end{equation}
One can use cubic convolutional interpolation \citep{keys1981cubic}, but higher
order methods lead to greater accuracy, and we therefore use quintic
interpolation~\cite{meijering1999image}.  The resulting D\hyp{}SKI kernel matrix
has the form
\begin{equation}\label{eqn:dski}
  \begin{bmatrix} K & (\dx K)^T \\ \dx K & \ddx K \end{bmatrix}  \approx
  \begin{bmatrix} W \\ \dx W  \end{bmatrix}  K_{UU}
  \begin{bmatrix} W \\ \dx W  \end{bmatrix} ^T =
  \begin{bmatrix} W\K{UU}W^T & W\K{UU} (\dx W)^T \\
  (\dx W) \K{UU} W^T & (\dx W) \K{UU} (\dx W)^T  \end{bmatrix}\,,
\end{equation}
where the elements of sparse matrices $W$ and  $\dx W$ are determined by $w_i
(x)$and $\nabla w_i(x)$ --- assuming quintic interpolation, $W$ and $\dx W$ will
each have $6^d$ elements per row. As with SKI, we use FFTs to obtain $
\calO(M \log M)$ MVMs with $\K{UU}$. Because $W$ and $\dx W$ have $\calO(N6^d)$
and $\calO(Nd6^d)$nonzero elements, respectively, our MVM complexity is $\calO
(Nd6^d + M \log M)$.

\subsection{D-SKIP}

Several common kernels are {\em separable}, i.e., they can be expressed as
products of one\hyp{}dimensional kernels.  Assuming a compatible approximation
scheme, this structure is inherited by the SKI approximation for the kernel
matrix without derivatives,
\begin{equation}\label{eqn:ski_sep}
  K \approx (W_1 K_1 W_1^T) \odot (W_2 K_2 W_2^T) \odot \ldots \odot (W_d K_d
  W_d^T)\,,
\end{equation}
where $A \odot B$ denotes the Hadamard product of matrices $A$ and $B$ with the
same dimensions, and $W_j$ and $K_j$ denote the SKI interpolation and inducing
point grid matrices in the $j$\hyp{}th coordinate direction. The same Hadamard
product structure applies to the kernel matrix with derivatives; for example,
for $d = 2$,
\begin{equation} \label{eqn:skip}
  \resizebox{.9\hsize}{!}{
    $K^{\nabla} \approx
    \begin{bmatrix}
        W_1 K_1 W_1^T      & W_1 K_1 \dx W_1^T     & W_1 K_1 W_1^T \\
        \dx W_1 K_1 W_1^T  & \dx W_1 K_1 \dx W_1^T & \dx W_1 K_1 W_1^T \\
        W_1 K_1 W_1^T      & W_1 K_1 \dx W_1^T     & W_1 K_1 W_1^T \\
    \end{bmatrix}
    \odot
    \begin{bmatrix}
        W_2 K_2 W_2^T      & W_2 K_2 W_2^T      & W_2 K_2 \dx W_2^T \\
        W_2 K_2 W_2^T      & W_2 K_2 W_2^T      & W_2 K_2 \dx W_2^T \\
        \dx W_2 K_2 W_2^T  & \dx W_2 K_2 W_2^T  & \dx W_2 K_2 \dx W_2^T \\
    \end{bmatrix}$
  }\,.
\end{equation}

\cref{eqn:skip} expresses $K^{\nabla}$ as a Hadamard product of one dimensional
kernel matrices. Following this approximation, we apply the SKIP reduction~
\citep{gardner2018product} and use Lanczos to further approximate 
\cref{eqn:skip} as $(Q_1 \Gamma_1 Q_1^T) \odot (Q_2 \Gamma_2 Q_2^T)$. This can
be used for fast MVMs with the kernel matrix. Applied to kernel matrices with
derivatives, we call this approach D\hyp{}SKIP.

Constructing the D\hyp{}SKIP kernel costs $\calO(d^2(N + M \log M+ r^3 N \log
d))$, and each MVM costs $\calO(d r^2 N)$ flops where $r$ is the effective rank
of the kernel at each step (rank of the Lanczos decomposition). We achieve high
accuracy with $r \ll N$. 

\subsection{Preconditioning}

Recent work \citep{cutajar2016preconditioning} has explored several 
preconditioners for exact kernel matrices. We have had success with
preconditioners of the form $H = \sigma^2 I + FF^T$ where $K^\nabla \approx
FF^T$ with $F \in \BBR^{N \times r}$. Solving with the
Sherman\hyp{}Morrison\hyp{}Woodbury formula ({\em a.k.a} the matrix inversion
lemma) is inaccurate for small $\sigma$; we use the more stable formula $H^{-1}
b = \sigma^{-2} (f-Q_1 (Q_1^T f))$ where $Q_1$ is computed in $\calO(r^2N)$ time
by the economy QR factorization
\begin{equation}\label{eqn:qrsmw}
  \begin{bmatrix} Q_1 \\ Q_2 \end{bmatrix} =
  \begin{bmatrix} F \\ \sigma I \end{bmatrix} R.
\end{equation}
In our experiments with solvers for D\hyp{}SKI and D\hyp{}SKIP, we have found
that a truncated pivoted Cholesky factorization, $K^{\nabla} \approx (\Pi L)(\Pi
L)^T$ works well for the low\hyp{}rank factorization. Computing the pivoted
Cholesky factorization is cheaper than MVM\hyp{}based preconditioners such as
Lanczos or truncated eigendecompositions as it only requires the diagonal and
the ability to form the rows where pivots are selected. Pivoted Cholesky is a
natural choice when inducing point methods are applied as the pivoting can
itself be viewed as an inducing point method where the most important
information is selected to construct a low\hyp{}rank preconditioner
\cite{harbrecht2012low}. The D\hyp{}SKI diagonal can be formed in $\calO(Nd6^d)$
flops while rows cost $\calO(Nd6^d + M)$ flops; for D\hyp{}SKIP both the
diagonal and the rows can be formed in $\calO(Nd)$ flops.

\subsection{Dimensionality Reduction}

In many high\hyp{}dimensional function approximation problems, only a few
directions are relevant. That is, if $f \Colon \BBR^D \to \BBR$ is a function
to be approximated, there is often a matrix $P$ with $d < D$ orthonormal columns
spanning an {\em active subspace} of $\BBR^D$ such that $f(x) \approx f(PP^T x)$
for all $x$ in some domain $\Omega$ of interest~\citep{constantine2015active}. 
The optimal subspace is given by the dominant eigenvectors of the covariance
matrix $C = \int_\Omega \nabla f(x) \, \nabla f(x)^T \, dx$, generally estimated
by Monte Carlo integration. Once the subspace is determined, the function can be
approximated through a GP on the reduced space, i.e., we replace the original
kernel $k(x,x')$ with a new kernel $\check k(x,x') = k(P^T x,P^T x')$.
Because we assume gradient information, dimensionality reduction based on active
subspaces is a natural pre\hyp{}processing phase before applying D\hyp{}SKI and
D\hyp{}SKIP.