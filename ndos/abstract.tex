Spectral analysis connects graph structure to the eigenvalues and eigenvectors
of associated matrices.  Much of spectral graph theory descends directly from
spectral geometry, the study of differentiable manifolds through the spectra of
associated differential operators.  But the translation from spectral geometry
to spectral graph theory has largely focused on results involving only a few
extreme eigenvalues and their associated eigenvalues.  Unlike in geometry, the
study of graphs through the overall distribution of eigenvalues --- the {\em
spectral density} --- is largely limited to simple random graph models.  The
interior of the spectrum of real-world graphs remains largely unexplored,
difficult to compute and to interpret.

In this chapter, we delve into the heart of spectral densities of real-world
graphs.  We borrow tools developed in condensed matter physics, and add novel
adaptations to handle the spectral signatures of common graph motifs.  The
resulting methods are highly efficient, as we illustrate by computing spectral
densities for graphs with over a billion edges on a single compute node. Beyond
providing visually compelling fingerprints of graphs, we show how the 
estimation of spectral densities facilitates the computation of many common
centrality measures, and use spectral densities to estimate meaningful
information about graph structure that cannot be inferred from the extremal
eigenpairs alone.