\subsection{KPM Approximation Error}
This section provides an error bound for our regularized DOS approximation
$K_\sigma\ast\mu$. We will start with the following theorem.
\begin{theorem}[\bf{Jackson's Theorem~\cite{jackson1911genauigkeit}}]
\label{thm:Jackson_damping}
	If $f:[-1,1]\rightarrow \BBR$ is Lipschitz continuous with constant $L$, its
	best degree $M$ polynomial approximation $\hat{f}^M$ has an $L_\infty$ 
	error of at most $6L/M$. The approximation can be constructed as
	\begin{equation}
		\hat{f}^M = \sum_{m=0}^MJ_mc_mT_m(x)\,,
	\end{equation}
	where $J_m$ are Jackson smoothing factors and $c_m$ are the Chebyshev
	coefficients.
\end{theorem}
We can pick a smooth mollifier $K$ with $\text{Lip}(K)=1$. For any $\nu\in\BBR$
and $\lambda\in [-1,1]$ there exists a degree $M$ polynomial such that 
\begin{equation}
	\abs{K_\sigma(\nu-\lambda)-\widehat{K}^{M}_\sigma(\nu-\lambda)} < \frac{6L}
	{M\sigma}\,.
\end{equation}
Define $\hat{\mu}^M = \sum_{m=0}^MJ_md_m\phi_m$ to be the truncated DOS
series,
\begin{equation}
	\int_{-1}^1 \hat{f}^M(\lambda)\mu(\lambda)d\lambda = \int_{-1}^1f
	(\lambda)\hat{\mu}^M(\lambda)d\lambda = \sum_{m=0}^MJ_mc_md_m\,.
\end{equation}
Therefore,
\begin{align*}
	\norm{K_\sigma\ast(\mu-\hat{\mu}^M)}_\infty
	=& \max_\nu\abs{\int_{-1}^1K_\sigma(\nu-\lambda)(\mu(\lambda)-\hat{\mu}^M 
	(\lambda))d\lambda}\\
	\leq& \max_\nu\int_{-1}^1\abs{K_\sigma(\nu-\lambda)-\widehat{K}^M_\sigma
	(\nu-\lambda)}\mu(\lambda)d\lambda \\
	\leq& \frac{6L}{M\sigma}\,.
\end{align*}
Consider $\tilde{\mu}^M$ to be the degree $M$ approximation from KPM,
\begin{equation}
	\norm{K_\sigma\ast(\mu-\tilde{\mu}^M)}_\infty \leq \norm{K_\sigma\ast(\mu- 
	\hat{\mu}^M)}_\infty+ \norm{K_\sigma}_\infty\norm{\,\hat{\mu}^M - 
	\tilde{\mu}^M}_1\,.
\end{equation}
If we use a probe $z$ with independent standard normal entries for the
trace estimation,
\begin{equation}
	\tilde{\mu}(\lambda) = \sum_{i=1}^Nw_i^2\delta(\lambda-\lambda_i)
\end{equation}
where $w=Q^Tz$ is the weight for $z$ in the eigenbasis. Hence
\begin{equation}
	\norm{\hat{\mu}^M-\tilde{\mu}^M}_1 \leq \sum_{i=1}^N \abs{1-w_i^2}\,.
\end{equation}
Finally,
\begin{equation}
	\BBE\left[\norm{K_\sigma\ast(\mu-\tilde{\mu}^M)}\right]\leq \frac{1}
	{\sigma}\left(\frac{6L}{M}+\norm{K}_\infty\BBE\left[
	\abs{1-w_1^2}\right]\right)\,.
\end{equation}
If we take $N_z$ independent probe vectors, then $N_zw_1^2\sim\chi^2(N_z)$,
which means the expectation decays asymptotically like $\sqrt{2 / (\pi N_z)}$.

\subsection{Perturbation Analysis}
In this section, we limit our attention to symmetric graph matrix $H$.
Extracting graph information using DOS, whether as a distribution for functions
on a graph or as a direct feature in the form of spectral moments, requires
stability under small perturbations. In the case of removing/adding a few 
number of nodes/edges, the Cauchy Interlacing Theorem~\cite{magnus1988matrix}
gives a bound on each individual new eigenvalue by the old ones. For example, 
if we remove $r\ll N$ nodes to get a new graph matrix $
\widetilde{H}$, then
\begin{equation}\label{eqn:interlacing}
	\lambda_i(H)\leq \lambda_i(\widetilde{H})\leq \lambda_{i+r}(H)\quad\text{for
	}\quad i\leq N-r\,.
\end{equation}
However, this bound may not be helpful when the impact of the change is not
reflected by its size. Hence, we provide a theorem that relates the Wasserstein
distance (see \cref{eqn:waisserstein}) change and the Frobenius norm of
the perturbation. Without loss of generality, we assume the eigenvalues of $H$
lie in $[-1,1]$ already.

\begin{theorem}
	Suppose $\widetilde{H} = H + \delta H$ is the perturbed graph matrix with
	spectral density $\tilde{\mu}$, then
	\begin{equation*}
		W_1(\mu, \tilde{\mu}) \leq \norm{\delta H}_F
	\end{equation*}
\end{theorem}
\begin{proof}
	Let $\calL$ be the space of Lipschitz functions with $f(0)=0$.
	\begin{align*}\label{eqn:proof1}
		W_1(\mu, \tilde{\mu})=& \sup_{f\in\calL, \Lip(f)=1} \int f(\lambda) (\mu
		(\lambda)-\tilde{\mu}(\lambda))d\lambda \\
		=&\frac{1}{N}\sup_{f\in\calL, \Lip(f)=1}\tr(f(H)-f(\widetilde{H}))\\
		\leq& \sup_{f\in\calL, \Lip(f)=1, \|v\|=1} v^T(f(H)-f(\widetilde{H}))v\,.
	\end{align*}
	By Theorem 3.8 from Higham~\cite{higham2008functions}, the perturbation on
	$f(H)$ is bounded by the Fr\'{e}chet derivative,
	\begin{equation}
		\norm{f(H) - f(\widetilde{H})}_2\leq \Lip(f)\|\delta H\|_F + \calo(\|\delta
		H\|_F).
	\end{equation}
\end{proof}