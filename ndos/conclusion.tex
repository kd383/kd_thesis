In this chapter, we make the computation of spectral densities a practical tool
for the analysis of large real-world network. Our approach borrows from methods
in solid state physics, but with adaptations that improve performance in the
network analysis setting by special handling of graph motifs that leave
distinctive spectral fingerprints. We show that the spectral densities are
stable to small changes in the graph, as well as providing an analysis of the
approximation error in our methods. We illustrate the efficiency of our
approach by treating graphs with tens of millions of nodes and billions of edges
using only a single compute node. The method provides a compelling visual
fingerprint of a graph, and we show how this fingerprint can be used for tasks
such as model verification.

Our approach opens the door for the use of complete spectral information in
large-scale network analysis. It provides a framework for scalable computation
of quantities already used in network science, such as common centrality
measures and graph connectivity indices (such as the Estrada index) that can be
expressed in terms of the diagonals and traces of matrix functions. But we
expect it to serve more generally to define new families of features that
describe graphs and the roles nodes play within those graphs. We have shown that
graphs from different backgrounds demonstrate distinct spectral
characteristics, and thus can be clustered based on those. Looking at LDOS
across nodes for role discovery, we can identify the ones with high similarity
in their local structures. Moreover, extracting nodes with large weights at
various points of the spectrum uncovers motifs and symmetries. In the future, we
expect to use DOS/LDOS as graph features for applications in graph clustering,
graph matching, role classification, and other tasks.

\noindent\textbf{Acknowledgments.} We thank NSF DMS-1620038 for supporting this
work.