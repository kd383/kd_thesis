Today massive datasets appear in countless applications across scientific
fields. Empowered by the the modern computational infrastructure, researchers
have spent tremendous effort in the past few decades to take advantage of the
information explosion. Machine learning, as one of the fields that rely heavily
on large data, has witnessed incredible development and had far-reaching impact
on nearly every area of scientific discovery. Even in traditional fields, such
as partial differential equations, people are working on larger and more complex
systems, hence more sizable datasets. Typically, matrices arise as the natural
representation of data in both explicit and implicit ways. The most common form
is the object\hyp{}feature matrices; there are also co-occurrence matrices and
kernel matrices capturing the relationship between pairs of objects; sometimes
they represent linear operators such as Laplacian matrices. In most instances,
the matrices grow in size with the datasets, so handling them becomes an
inevitable challenge. Thus, numerical linear algebra (NLA), the study of 
matrix\hyp{}based numerical methods, promptly becomes the driving force behind
a lot of these applications.

Similarly, this ongoing trend of big data has led NLA into new and exciting
directions. Traditionally, NLA has focused on deterministic, 
factorization\hyp{}based methods. While they usually have strong theoretical
guarantees and highly optimized implementations, the emerging problems have far
exceeded the scale they are designed for, even with present\hyp{}day
computational power. As a result, a great number of novel algorithms
centered around scalability are developed in response to this increasing demand.
The class of randomized methods is one particular approach that brings
efficiency through introducing stochasticity. Randomized NLA aims at building
fast algorithms that return sufficiently good approximation to solutions. Both
``fast'' and ``good'' are measured relatively depending on the underlying
problem, but they are usually supported by matrix perturbation theory and
probability theory. Even though scalability is the foremost objective,
randomized NLA also designs algorithms around other criteria, e.g.
interpretability, robustness. \citet{drineas2016randnla} give a nice overview
for the recent development in randomized NLA.

This dissertation applies randomized NLA to large-scale matrix data in two
settings:
\begin{enumerate}
	\item The matrix contains low\hyp{}order information compared to its size,
	such as low\hyp{}rank matrix and structured matrix. The goal is to efficiently
	obtain a compressed representation of the matrix, ideally in a interpretable
	way.
	\item Only partial information from the matrix is required. The matrix
	could be explicitly available, but the extraction of information is
	expensive; or the matrix is implicit, due to the cost of formation or
	storage.
\end{enumerate}

It is worth mentioning that these two scenarios are not mutually exclusive.
For example, \cref{ch4} covers the computation of log\hyp{}determinant for
kernel matrices, where we can often take advantage of their low\hyp{}rank plus
diagonal structure. The applications we work with come from network science,
Gaussian processes regression, natural language processing, and quantum
chemistry. The diversity in background alone is a compelling evidence for the
versatility of randomized NLA. The outline is detailed below.

\cref{ch2} covers the mathematical preliminaries as well as the common topics
shared between the following chapters.

\cref{ch3} studies the spectral densities of massive real-world graphs. We
borrow tools developed in condensed matter physics, and add novel adaptations to
handle the spectral signatures of common graph motifs.  The resulting methods
are highly efficient, as we illustrate by computing spectral densities for
graphs with over a billion edges on a single compute node. Beyond providing
visually compelling fingerprints of graphs, we show how the  estimation of
spectral densities facilitates the computation of many common centrality
measures, and use spectral densities to estimate meaningful information about
graph structure that cannot be inferred from the extremal eigenpairs alone.

\cref{ch4} develop novel approaches for Gaussian processes (GPs) regression.
The computational bottleneck of kernel learning for GPs is computing the log
determinant and its derivatives of an $N\Times N$ kernel matrix. Building on
existing fast matrix-vector-multiplication approximation for kernel
matrices, we combine iterative methods with stochastic estimation to
lower the cost from $\calO(N^3)$ to $\calO(N)$. The resulting methods are highly
efficient and flexible, allowing us to work with datasets much larger than
the traditional GP capability. Furthermore, we extend these ideas to GP
regression on both function values and derivatives. Our approaches, together
with dimensionality reduction and preconditioning, let us scale Bayesian
optimization with derivatives to high-dimensional problems and large evaluation
budgets.

\cref{ch5} proposes a complete pipeline for spectral inference of topic
models that scales gracefully with both the size of vocabulary and the dimension
of latent space. It allows us to simultaneously compress and rectify the 
co\hyp{}occurrence statistics, then learn latent variables directly from the
compressed form without losing visible precision. We verify that our
methods perform comparably to previous approaches on both textual and
non-textual data.

\cref{ch6} describes how to use centroid Voronoi tessellation to accelerate
electronic structure calculation. The recently developed interpolative separable
density fitting decomposition compresses the redundant information in electron
orbital pairs through a set of non-uniform interpolation points. Our method,
implemented as a weight K-means algorithm with random initialization, achieves
comparable accuracy to the existing procedure but at a cost negligible in the
overall calculations. We also find that our algorithm as a continuation method
enhances the smoothness of the potential energy surface.