The rapid growth of data brings brand new opportunities and challenges
to researchers across many fields, e.g., mathematics, computer science,
statistics, scientific computing, and machine learning. There is more interest
than ever in scalable numerical algorithms that take advantage of the abundant
data. Randomized numerical linear algebra provides a powerful
set of tools for handling large\hyp{}scale matrix data. In this dissertation, we
have presented a collection of randomized NLA algorithms for overcoming the
computational obstacles in our problems. The applications we work with come from
diverse areas, including network science, Gaussian process regression, natural
language processing, and quantum chemistry. In all these cases, randomized NLA
has led to efficient and robust algorithms, enabling us to extract valuable
information from massive datasets. As the ongoing trend of large data persists,
we expect randomized NLA to play a more and more important role in the
development of practical algorithms. This dissertation has also left many
interesting open questions, which we are excited to explore in the future.

In \cref{ch3} we developed stochastic approximation methods for the spectral
density of large real\hyp{}world graphs. We were able to link the spectral
characteristics of a graph to its structure properties. The spectral
fingerprint of networks offers a new family of features that can be utilized in
learning tasks. There are many directions available for future work. In this
chapter, we mainly focused on normalized graph adjacency/Laplacian because of
its popularity in graph partitioning. However, there exist many other graph
matrices, each of which represents a unique aspect. Inspecting different types
of graph matrices should provide new insights on graph structures. In addition,
we are looking to extend the error analysis of our methods. From a practical
perspective, we have yet demonstrated the full power of spectral information in
graph\hyp{}related applications. In particular, we hope to incorporate the
quantities we compute into graph clustering, graph classification, and node role
discovery.

In \cref{ch4} we proposed multiple methods to efficiently estimate the log
determinant of kernel matrices, thus scaling Gaussian process regression to
handle massive datasets. Our methods are highly flexible as they build upon the
fast matrix\hyp{}vector multiplication from existing kernel approximations. We
showcased this generality by extending our approach to include derivative
information. Together with dimensionality reduction through an active subspace
method, we achieved efficient Bayesian optimization in high\hyp{}dimensional
space. Our algorithms have been incorporated into the GPyTorch library, a Python
package for scalable Gaussian process regression. In the future, we look
forward to applying our methods in other scenarios involving GPs.

In \cref{ch5} we created a new scalable and robust framework for spectral
inference of topic models. Through this pipeline, we are able to simultaneous
compress and rectify large matrices of co\hyp{}occurrence statistics, which then
can be used for efficient inference of the underlying structure. Rectification,
as a key ingredient in this work, helps correct the mismatch between proposed
models and noisy data. The noise level in data varies significantly across
applications, some of which are known for having low signal\hyp{}noise ratio.
Therefore, we hope to apply our scalable rectification to those applications,
which will enhance the robustness of the existing methods. Moreover, the 
$\epsilon$\hyp{}nonnegativity is a rather heuristic approach toward our problem.
In the future, we hope to find a better way of detecting negative entries in the
outer product form $XX^T$.


In \cref{ch6} we used centroid Voronoi tessellation, implemented as weighted 
K\hyp{}means algorithm, to accelerate electronic structure calculation. As a
replacement for the deterministic counterpart, our method brings tremendous
speed\hyp{}up without any loss of accuracy in practice. Nonetheless, the
remaining computation is still very time\hyp{}consuming. Electronic structure
calculation, along with other quantum chemistry problems, are among the most
popular tasks on supercomputers. Therefore, it is very exciting to seek more
opportunities where randomized NLA can improve the efficiency of those
computation while maintaining high accuracy. On the other hand, there are many
more numerical methods in this field that, like the kernel polynomial method,
can be adapted to new applications. We hope to further explore these in the
future.