Orbital pairs of the form $\{\varphi_{i}(r)\psi_{j}(r)\}_{i,j=1}^{N}$, where
$\varphi_{i}, \psi_{j}$ are single particle orbitals, appear ubiquitously in
quantum chemistry. A few examples include the Fock exchange operator, the MP2
amplitude, and the polarizability operator \cite{SzaboOstlund1989,Martin2004}.
When $N$ is proportional to the number of electrons $N_{e}$ in the system, the
total number of orbital pairs is $N^2\Sim\calO(N_{e}^2)$. On the other
hand, the number of degrees of freedom needed to resolve all orbital pairs on a
dense grid is only $\calO(N_{e})$.  Hence as $N_{e}$ becomes large, the set of
all orbital pairs contains apparent redundant information. In order to compress
the redundant information and to design more efficient numerical algorithms,
many algorithms in the past few decades have been developed. Pseudospectral
decomposition \cite{MurphyJCP1995,ReynoldsJCP1996}, Cholesky decomposition 
\cite{BeebeLinderberg1977,KochSanchezdePedersen2003,AquilantePedersenLindh2007,
ManzerHornMardirossianEtAl2015}, density fitting (DF) or resolution of identity
(RI) \cite{NJP_14_053020_2012,Weigend2002}, and tensor hypercontraction (THC) 
\cite{ParrishHohensteinMartinezEtAl2012,ParrishHohensteinMartinezEtAl2013} are
only a few examples towards this goal. When the single particle orbitals
$\varphi_{i},\psi_{j}$ are already localized functions, ``local methods'' or
``linear scaling methods'' \cite{Goedecker1999, RPP_75_036503_2010_ON,
GuidonHutterVandevondele2010, CP_356_98_2009} can be applied to construct such
decomposition with cost that scales linearly with respect to $N_{e}$. Otherwise,
the storage cost of the matrix to represent all orbital pairs on a grid is
already $\calO(N_{e}^3)$, and the computational cost of compressing the orbital
pairs is then typically $\calO(N_{e}^{4})$.

Recently, \citet{JCP_302_329_2015_ISDF} developed a new decomposition called the
\emph{interpolative separable density fitting} (ISDF) , which takes the
following form
\begin{equation}\label{eqn:isdfformat}
  \varphi_{i}(r)  \psi_{j}(r)\approx \sum_{\mu=1}^{N_{\mu}}\zeta_{\mu}(r) \left(
  \varphi_{i}(\rhat_\mu)\psi_{j}(\rhat_\mu)\right)\,.
\end{equation}
For a given $r$, if we view $[\psi_{i}(r)\psi_{j}(r)]$ as a row of the matrix $
\{\psi_{i}\psi_{j}\}$ discretized on a dense grid, then the ISDF decomposition
states that all such matrix rows can be approximately expanded using a linear
combination of matrix rows with respect to a selected set of 
\textit{interpolation points} $\{\rhat_\mu\}_{\mu=1}^{N_{\mu}}$. The
coefficients of such linear combination, or \textit{interpolating vectors}, are
denoted by $\{\Bzeta_{\mu}(r)\}_{\mu=1}^{N_{\mu}}$. Here $N_{\mu}$ can be
interpreted as the numerical rank of the ISDF decomposition. Compared to the
standard density fitting method, the three-tensor $\left\{\varphi_{i}
(\rhat_\mu)\psi_{j}(\rhat_\mu)\right\}$ with three indices $i, j, \mu$ takes a
separable form. This reduces the storage cost of the decomposed tensor from
$\calO(N_{e}^3)$ to $\calO(N_{e}^2)$, and the computational cost from $\calO(N_
{e}^{4})$ to $\calO(N_{e}^3)$. Note that if the interpolation points $
\{\rhat_\mu\}_{\mu=1}^{N_{\mu}}$ are chosen to be on a uniform grid, then the
ISDF decomposition reduces to the pseudospectral decomposition, where $N_
{\mu}\Sim\calO(N_{e})$ but with a large preconstant. For instance, the
pseudospectral decomposition can be highly inefficient for molecular systems,
where the grid points in the vacuum contribute nearly negligibly to the orbital
pairs. On the other hand, by selecting the interpolation points carefully, e.g.
through a randomized \emph{QR factorization with column pivoting} (QRCP)
procedure \cite{GolubVan2013}, the number of interpolation points can be
significantly reduced. The QRCP based ISDF decomposition has been applied to
accelerate a number of applications, at least in the context of pseudopotential
approximation where the wavefunctions are smooth, including two\hyp{}electron
integral computation \cite{JCP_302_329_2015_ISDF}, correlation energy in the
random phase approximation \cite{LuThicke2017}, density functional perturbation
theory \cite{LinXuYing2017}, and hybrid density functional calculations 
\cite{JCTC_2017_ISDF}. For example, when iterative solvers are used for hybrid
density functional calculations, the Fock exchange operator $V_{X}$ defined in
terms of a set of orbitals $\{\varphi_i\}$ needs to be repeatedly applied to
another set of Kohn-Sham orbitals $\{\varphi_j\}$
\begin{equation}\label{eqn:applyVX}
  \left(V_{X}[\{\varphi_{i}\}]\psi_{j}\right)(r) = -\sum_{i=1}^{N_{e}} \varphi_
  {i}(r) \int K(r,r')\varphi_{i}(r')\psi_{j}(r') dr'.
\end{equation}
where $K(r, r')$ is the kernel for the Coulomb or the screened Coulomb
operator.  The integration in Eq. \ref{eqn:applyVX} is often carried out by
solving Poisson-like equations, using e.g. a fast Fourier transform (FFT)
method, and the computational cost is $\calO(N_e^3)$ with a large preconstant.
This is typically the most time consuming component in hybrid functional
calculations, and can be accelerated by the ISDF decomposition for the orbital
pairs $\{\varphi_{i}\psi_{j}\}$.

In Ref.~\cite{JCP_302_329_2015_ISDF}, the interpolation points and the
interpolation vectors are determined simultaneously through randomized QRCP
applied to $\{\psi_{i}(\Br)\psi_{j}(\Br)\}$ directly. We recently found that the
randomized QRCP procedure has $\calO(N_{e}^3)$ complexity but with a relatively
large preconstant, and may not be competitive enough when used repeatedly. In
order to overcome such difficulty, we proposed a different approach in 
Ref.~\cite{JCTC_2017_ISDF} that determines the two parts separately and reduces
the computational cost. We use the relatively expensive randomized QRCP
procedure to find the interpolation points in advance, and only recompute the
interpolation vectors whenever $\{\psi_{i}(r)\psi_{j}(r)\}$ has been updated
using an efficient least squares procedure that exploits the separable nature of
the matrix to be approximated. As a result, we can significantly accelerate
hybrid functional calculations using the ISDF decomposition in all but the first
\emph{self consistent field} (SCF) iteration.

In this work, we further remove the need of performing the QRCP decomposition
completely and, hence, significantly reduce the computational cost. Note that an
effective choice of the set of interpolation points should satisfy the following
two conditions. (1) The distribution of the interpolation points should roughly
follow the distribution of the electron density. In particular, there should be
more points when the electron density is high, and less or even zero points if
the electron density is very low. (2) The interpolation points should not be very
close to each other.  Otherwise, matrix rows represented by the interpolation
points are nearly linearly dependent, and the matrix formed by the interpolation
vectors will be highly ill\hyp{}conditioned.  The QRCP procedure satisfies both 
(1) and (2) simultaneously, and thus is an effective way for selecting the
interpolation points.  Here we demonstrate that (1) and (2) can also be satisfied
through a much simpler centroidal Voronoi tessellation (CVT) procedure applied
to a weight vector such as the electron density.

The Voronoi tessellation technique has been widely used in computer science
\cite{aurenhammer1991voronoi}, as well as scientific and engineering
applications, e.g., image processing\cite{du2006centroidal}, pattern
recognition \cite{ogniewicz1995hierarchic}, and numerical integration 
\cite{becke1988multicenter}. The concept of Voronoi tessellation can be simply
understood as follows. Given a discrete set of weighted points, the CVT
procedure divides a domain  into a number of regions, each consisting of a
collection of points that are closest to its weighted centroid. Here we choose
the electron density as the weight, and the centroids as the interpolation
points. The centroids must be located where the electron density is significant,
and hence satisfy the requirement (1). The centroids are also mutually separated
from each other by a finite distance due to the nearest neighbor principle, and
hence satisfy the requirement (2). Although detailed analysis of the error
stemming from such a choice of interpolation points is very difficult for
general nonlinear functions, we find that the CVT procedure approximately
minimizes the residual of the ISDF decomposition in \cref{eqn:isdfformat}.
In practice, the CVT procedure only applies to one vector (the electron density)
instead of $\calO(N_{e}^2)$ vectors and hence is very efficient.

We apply the ISDF\hyp{}CVT method to accelerate hybrid functional calculations
in a planewave basis set. We perform such calculations for different systems
with insulating (liquid water), semiconducting (bulk silicon), and metallic 
(disordered silicon aluminum alloy) characters, as well as ab initio molecular
dynamics (AIMD) simulations. We find that the ISDF\hyp{}CVT method achieves
similar accuracy to that obtained from the ISDF\hyp{}QRCP method, with
significantly improved efficiency. For instance, for a bulk silicon system
containing $1000$ silicon atoms computed on $2000$ computational cores with
kinetic energy cutoff being $10\Ha$, the QRCP procedure finds the interpolation
points with $38.1\Sec$, while the CVT procedure only takes $0.7\Sec$. Since the
solution of the CVT procedure is continuous with respect to changes in the
electron density, we also find that the CVT procedure produces a smoother
potential energy surface than that by the QRCP procedure in the context of AIMD
simulations.

The remainder of the chapter is organized as follows. We briefly introduce the
ISDF decomposition in \cref{cvtsec:bac}. In \cref{cvtsec:met} we describe
the ISDF\hyp{}CVT procedure and its implementation for hybrid functional
calculations. We present numerical results of the ISDF\hyp{}CVT method in 
\cref{cvtsec:num}, and conclude in \cref{cvtsec:con}.