In this work, we demonstrate that the interpolative separable density fitting
decomposition (ISDF) can be efficiently performed through a separated treatment
of interpolation points and interpolation vectors. We find that the centroidal
Voronoi tessellation method (CVT) provides an effective choice of interpolation
points using only the electron density as the input information. The resulting
interpolation points are by design inhomogeneous in the real space, concentrated
at regions where the electron density is significant, and are well separated
from each other. These are all key ingredients for obtaining a low rank
decomposition that is accurate and a well conditioned set of interpolation
vectors. We demonstrate that the CVT\hyp{}based ISDF decomposition can be an
effective strategy for reducing the cost hybrid functional calculations for
large systems. The CVT\hyp{}based method achieves similar accuracy when compared
with that obtained from QRCP, with significantly improved efficiency. Since the
solution of the CVT method depends continuously with respect to the electron
density, we also find that the CVT method produces a smoother potential energy
surface than that by the QRCP method in the context of ab initio molecular
dynamics simulation. Our analysis indicates that it might be possible to further
improve the quality of the interpolation points by taking into account the
gradient information in the weight vector. We also expect that the 
CVT\hyp{}based strategy can also be useful in other contexts where the ISDF
decomposition is applicable, such as ground state calculations with rung\hyp{}5
exchange\hyp{}correlation functionals, and excited state calculations. These
will be explored in the future work.