In this section, we briefly introduce the ISDF decomposition 
\cite{JCP_302_329_2015_ISDF} evaluated using the method developed in Ref. 
\cite{JCTC_2017_ISDF}, which employs a separate treatment of the interpolation
points and interpolation vectors.

First, assume the interpolation points $
\{\Brhat_\mu\}_{\mu=1}^{N_{\mu}}$ are known, then the interpolation vectors can
be efficiently evaluated using a least squares method as follows. Using a linear
algebra notation, Eq. \ref{eqn:isdfformat} can be written as
\begin{equation}\label{eqn:isdflineq}
  \BZ \approx \BTheta \BC,
\end{equation}
where each column of $\BZ$ is given by $\BZ_{ij}(\Br)=\varphi_{i}(\Br)\psi_{j}
(\Br)$ sampled on a dense real space grids $\{\Br_{i}\}_{i=1}^{N_g}$, and
$\BTheta = [\Bzeta_1, \Bzeta_2, ..., \Bzeta_{N_{\mu}}]$ contains the
interpolating vectors. Each column of $\BC$ indexed by $(i,j)$ is given by
\begin{equation}
\left[ \varphi_{i}(\Brhat_1)\psi_{j}(\Brhat_1),\; \cdots,\;
    \varphi_{i}(\Brhat_\mu)\psi_{j}(\Brhat_\mu),\; \cdots,\;
    \varphi_{i}(\Brhat_{N_{\mu}})\psi_{j}(\Brhat_{N_{\mu}})\right]^T.
\end{equation}
Eq. \ref{eqn:isdflineq} is an overdetermined linear system with respect to the
interpolation vectors $\BTheta$. The least squares approximation to the solution
is given by
\begin{equation}\label{eq:Theta}
  \BTheta = \BZ\BC^T (\BC\BC^T)^{-1}.
\end{equation}
It may appear that the matrix-matrix multiplications $\BZ\BC^T$ and $\BC\BC^T$
take $\calO(N_{e}^{4})$ operations because the size of $\BZ$ is $N_g \times N^2$
and the size of $\BC$ is $N_{\mu} \times N^2$.  However, both multiplications
can be carried out with fewer operations due to the separable structure of $\BZ$
and $\BC$. The computational complexity for computing the interpolation vectors
is $\calO(N_{e}^{3})$, and numerical results indicate that the preconstant is
also much smaller than that involved in hybrid functional calculations 
\cite{JCTC_2017_ISDF}. Hence the interpolation vectors can be obtained
efficiently using the least squares procedure.

The problem for finding a suitable set of interpolation points $\{\Brhat_\mu\}_
{\mu=1}^{N_{\mu}}$ can be formulated as the following linear algebra problem.
Consider the discretized matrix $\BZ$ of size $N_{g}\times N^2$, and find $N_
{\mu}$ rows of $\BZ$ so that the rest of the rows of $\BZ$ can be approximated
by the linear combination of the selected $N_{\mu}$ rows. This is called an
interpolative decomposition \cite{SIAM_13_727_1992_QRCP}, and a standard method
to achieve such a decomposition is the QR factorization with column pivoting 
(QRCP) procedure \cite{SIAM_13_727_1992_QRCP} as
\begin{equation}\label{eqn:QRCP}
  \BZ^{T} \BPi = \BQ\BR.
\end{equation}
Here $\BZ^T$ is the transpose of $\BZ$, $\BQ$ is an $N^2 \times N_g$ matrix that
has orthonormal columns, $\BR$ is an upper triangular matrix, and $\BPi$ is a
permutation matrix chosen so that the magnitude of the diagonal elements of
$\BR$ form an non-increasing sequence.  The magnitude of each diagonal element
$\BR$ indicates how important the corresponding column of the permuted $\BZ^T$
is, and whether the corresponding grid point should be chosen as an
interpolation point. The QRCP factorization can be terminated when the $(N_
{\mu}+1)$-th diagonal element of $\BR$ becomes less than a predetermined
threshold. The leading $N_{\mu}$ columns of the permuted $\BZ^T$ are considered
to be linearly independent numerically. The corresponding grid points are chosen
as the interpolation points. The indices for the chosen interpolation points $
\{\Brhat_\mu\}$ can be obtained from indices of the nonzero entries of the first
$N_{\mu}$ columns of the permutation matrix $\BPi$.

The QRCP decomposition satisfies the requirements (1) and (2) discussed in \S
\ref{c5sec:int}. First, QRCP permutes matrix columns of $\BZ^{T}$ with large
norms to the front, and pushes matrix columns of $\BZ^{T}$ with small norms to
the back. Note that the square of the vector 2\hyp{}norm of the column of $\BZ^
{T}$ labeled by $\Br$ is just
\begin{equation}\label{eqn:Znorm}
  \sum_{i,j=1}^{N} \varphi_{i}^2(\Br) \varphi_{j}^2(\Br) =
  \left(\sum_{i=1}^{N} \varphi_{i}^2(\Br)\right) \left(\sum_{j=1}^{N}
  \psi_{j}^2(\Br)\right).
\end{equation}
In the case when $\varphi_{i},\psi_{j}$ are the set of occupied orbitals, the
norm of each column of $\BZ^{T}$ is simply the electron density. Hence the
interpolation points chosen by QRCP will occur where the electron density is
significant. Second, once a column is selected, all other columns are
immediately orthogonalized with respect to the chosen column. Hence nearly
linearly dependent matrix columns will not be selected repeatedly. As a result,
the interpolation points chosen by QRCP are well separated spatially.

It turns out that the direct application of the QRCP procedure (Eq. 
\ref{eqn:QRCP}) still requires $\calO(N_{e}^{4})$ computational complexity.  The
key idea used in Ref. \cite{JCP_302_329_2015_ISDF} to lower the cost is to
randomly subsample columns of the matrix $\BZ$ to form a smaller matrix
$\BZtilde$ of size $N_{g}\times \widetilde{N}_\mu$, where $\widetilde{N}_\mu$ is
only slightly larger than $N_{\mu}$.  Applying the QRCP procedure to this
subsampled matrix $\BZtilde$ approximately yields the choice of interpolation
points, but the computational complexity is reduced to $\calO(N_{e}^3)$. In the
context of hybrid density functional calculations, the cost of the randomized
QRCP method can be comparable to that of applying the exchange operator in the
planewave basis set \cite{JCTC_2017_ISDF}. However, the ISDF decomposition can
still significantly reduce the computational cost, since the interpolation
points only need to be performed once for a fixed geometric configuration.