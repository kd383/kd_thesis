The recently developed interpolative separable density fitting (ISDF)
decomposition is a powerful way for compressing the redundant information in the
set of orbital pairs, and has been used to accelerate quantum chemistry
calculations in a number of contexts. The key ingredient of the ISDF
decomposition is to select a set of non-uniform grid points, so that the values
of the orbital pairs evaluated at such grid points can be used to accurately
interpolate those evaluated at all grid points. The set of non-uniform grid
points, called the interpolation points, can be automatically selected by a QR
factorization with column pivoting (QRCP) procedure. This is the computationally
most expensive step in the construction of the ISDF decomposition. In this work,
we propose a new approach to find the interpolation points based on the
centroidal Voronoi tessellation (CVT) method, which offers a much less expensive
alternative to the QRCP procedure when ISDF is used in the context of hybrid
functional electronic structure calculations. The CVT method only uses
information from the electron density, and can be efficiently implemented using
a K-Means algorithm. We find that this new method achieves comparable accuracy
to the ISDF-QRCP method, at a cost that is negligible in the overall hybrid
functional calculations. For instance, for a system containing $1000$ silicon
atoms simulated using the HSE06 hybrid functional on $2000$ computational cores,
the cost of QRCP-based method for finding the interpolation points costs
$38.1\Sec$, while the CVT procedure only takes $0.7\Sec$.  We also find that the
ISDF-CVT method also enhances the smoothness of the potential energy surface in
the context of \emph{ab initio} molecular dynamics (AIMD) simulations with
hybrid functionals.