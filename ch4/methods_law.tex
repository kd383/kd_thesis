The output for both methods in \S \ref{c4sec:lrr} is a compressed co
\hyp{}occurrence matrix $\BC \Eq \BY\BY^T$. In this section, we present the 
\textbf{Low-rank Anchor Word algorithm (LAW)} that reduces the time complexity
of finding anchor objects from $\calO(N^2K)$ to $\calO(NK^2)$ by taking
advantage of this form. We note that LAW applies whenever $\BC$ is in a low-rank
representation, which does not have to be derived from our methods. Moreover, it
is exact for non-negative $\BC$, but it is robust in practice when we allow
small negative entries in $\BC$, as in the case with ENN.

The first step is to $L_1$\hyp{}normalize the rows of $\BC$. Given $\BC \Geq 0$,
the $L_1$\hyp{}norm of each row is simply the sum of all its entries, so we can
calculate the row norms by $\Bd \Eq \BY(\BY^T\Be)$. To obtain the normalized
$\BC$, we simply scale the rows of $\BY$, and \smash{$\BCbar \Eq (\diag(\Bd)^
{-1}\BY)\BY^T \Eq \BYbar\BY^T$}. These steps cost $\calO(NK)$.

Next, we need to apply column pivoted QR to \smash{$\BCbar^T$} in order to
identify the pivots as our anchor objects $\BS$. By taking the QR decomposition
$\BY \Eq \BQ\BR$, $\BCbar^T$ can be further transformed into $\BQ\BR\BYbar^T$.
Notice that $\BCbar^T$ is an orthogonal embedding of $\BX^T = \BR\BYbar^T$ onto
a higher-dimensional space, which preserves the column $L_2$-norms. Lemma 
\ref{lem:qrequiv} shows that column pivoted QR on $\BCbar^T$ and on
$\BR\BYbar^T$ are equivalent, which allows us to lower the computation cost from
$\calO(N^2K)$ to $\calO(NK^2)$.
\vspace*{\fill}
\begin{figure}[h]
	\begin{algorithm}[H]
		\SetKwInput{KwInput}{Input}
		\SetKwInput{KwOutput}{Output}
		\DontPrintSemicolon
    \KwInput{Object co-occurrence $\BC \Eq \BY \BY^T$}
		\KwOutput{Anchor objects $\BS \Eq \{s_1, ..., s_K\}$\\ 
		\hspace{36px} Latent clusters $\BB \In \BBR^{N \Times K}$\\
		\hspace{36px} Cluster correlations $\BA \In \BBR^{K \Times K}$}
		\Begin{
			Calculate row sums $\Bd = \BY(\BY^T\Be)$.\;
			Normalize $\BYbar \leftarrow \diag(\Bd)^{-1}\BY$.\;
			\vspace{3px}
			Compute QR decomposition of $\BY = \BQ\BR$.\;
			Form $\BX = \BYbar\BR^T$.\;
			Select $\BS$ using column pivoted QR on $\BX^T$.\;
			\vspace{2px}
			Solve $n$ simplex-constrained least square problems
			to minimize $\|\BX-\BBreve\BX_{\BS\ast}\|_F$.\;
			\vspace{2px}
			Recover $\BB$ from $\BBreve$ using Bayes' rule.\;
			Recover $\BA = \BB_{\BS \ast}^{-1}\BY_{\BS\ast}\BY_{\BS\ast}^T\BB_{\BS 
			\ast}^{-1}$.
		}
		\caption{Low-rank AW (LAW)}
	  \label{alg:law}      
	\end{algorithm}
	\vspace{2cm}
	\begin{algorithm}[H]
		\SetKwInput{KwInput}{Input}
		\SetKwInput{KwOutput}{Output}
		\DontPrintSemicolon
    \KwInput{Raw object-example $\BH \In \BBR^{N \Times M}$}
		\KwOutput{Anchor objects $\BS \Eq \{s_1, ..., s_K\}$\\ 
		\hspace{36px} Latent clusters $\BB \In \BBR^{N \Times K}$\\
		\hspace{36px} Cluster correlations $\BA \In \BBR^{K \Times K}$}
		\Begin{
		  Get $\BHhat, \BHhat_{diag}$ from $\BH$ by \eqref{eqn:unbiased}.\;
		  \vspace{1px}
			$\BC_{op}: \Bx\rightarrow\BHhat (\BHhat^T\Bx) - \BHdiag\Bx$\;
			\vspace{3px}
			$(\BU,\BLambda_K) \leftarrow$ Randomized-Eig($\BC_{op}, K$)\;
			\vspace{3px}
			Initialize ENN with $\BU, \BLambda_K$.\;
			\vspace{3px}
			$\BY\leftarrow$ ENN-rectification\;
			\vspace{3px}
			$(\BS, \BB, \BA) \gets $LAW($\BY)\;\;$ (Algorithm \ref{alg:law})
		}
		\caption{Low-rank JSMF (LR-JSMF)}
	\label{alg:lr-jsmf}        
	\end{algorithm}
\end{figure}
\FloatBarrier

\begin{lemma}\label{lem:qrequiv}
  Let $\BS$ be the set of pivots that have been selected by column pivoted QR on
  $\BCbar^T = \BQ\BX^T$. Given the QR decomposition, $\BXbar_{\BS\ast}^T = \BP
  \BT$, then $\BCbar_{\BS\ast}^T = (\BQ\BP)\BT$ is the corresponding QR
  decomposition for the columns of $\BCbar$. For any remaining row $i\in 
  [N]\setminus S$,
  \begin{equation}
  	\|(I-\BP\BP^T)\BXbar_{i\ast}^T\|_2 = \|(I-(\BQ\BP)(\BQ\BP)^T)\BCbar_
  	{i\ast}^T\|_2
  \end{equation}
  Therefore, the next column pivot is identical for $\BCbar^T$ and $\BXbar^T$. 
  By induction, column pivoted QR on $\BCbar^T$ and $\BXbar^T$ return the same
  pivots.
\end{lemma}
\begingroup
\allowdisplaybreaks
\begin{proof}
  Because both $\BQ$ and $\BP$ have orthonormal columns,
  \begin{equation}
    (\BQ\BP)^T(\BQ\BP) = \BP^T(\BQ^T\BQ)\BP = \BP^T\BP = I
  \end{equation}
  Thus, $\BQ\BP$ and $\BT$ forms the QR decomposition of $\BCbar_{\BS\ast}^T$.
  The residual of a remaining column $i\in[N]\setminus S$ is $(I-(\BQ\BP)(\BQ
  \BP)^T) \BCbar_{i\ast}^T$ and $(I-\BP\BP^T)\BXbar_{i\ast}^T$ for $\BCbar^T$
  and $\BXbar^T$, respectively. Simplify the former gives us
  \begin{align*}
    &(I-(\BQ\BP)(\BQ\BP)^T)\BCbar_{i\ast}^T \\
    ={} & (I-(\BQ\BP)(\BQ\BP)^T)\BQ\BXbar_{i\ast}^T \\
    ={} & \BQ\BXbar_{i\ast}^T - \BQ\BP\BP^T\BQ^T\BQ\BXbar_{i\ast}^T\\
    ={} & \BQ(I-\BP\BP^T)\BXbar_{i\ast}^T
  \end{align*}
  Finally,
  \begin{equation}\label{eqn:normpres}
    \|(I-(\BQ\BP)(\BQ\BP)^T)\BCbar_{i\ast}^T\|_2^2 = \BXbar_{i\ast}(I-\BP\BP^T)
    \BQ^T\BQ(I-\BP\BP^T)\BXbar_{i\ast}^T = \|(I-\BP\BP^T)\BXbar_{i\ast}^T\|_2^2
  \end{equation}
  Because the next pivot is selected as the column whose residual has the
  largest $L_2$\hyp{}norm, Eq. \ref{eqn:normpres} indicates that the same pivot
  will be selected for $\BCbar^T$ and $\BXbar^T$. Inductively, the anchors $\BS$
  recovered by column pivoted QR on those matrices are equivalent.
\end{proof}
\endgroup

Following the recovery of $\BS$, AW solves $N$ independent simplex-constrained
least square problems $\|\BCbar_{i\ast} - \BBreve_{i\ast}^T\BCbar_
{\BS\ast}\|_2$. Again we can leverage the $L_2$-norm preserving property, 
\begin{equation}\label{eqn:lrls}
  \|\BCbar_{i\ast} - \BBreve_{i\ast}^T\BCbar_{\BS\ast}\|_2 = \|\BX_{i\ast}\BQ^T
  - \BBreve_{i\ast}^T\BX_{\BS\ast}\BQ^T\|_2 = \|\BX_{i\ast} - \BBreve_{i\ast}^T 
  \BX_{\BS\ast}\|_2
\end{equation}
and reduce the dimension of the least\hyp{}square problems from $N$ to $K$,
thereby the complexity from $\calO(N^2K)$ to $\calO(NK^2)$. The remaining part
of the algorithm follows exactly as AW.

\paragraph{Low\hyp{}rank Joint Stochastic Matrix Factorization (LR-JSMF)} 
We complete our scalable framework of processing co\hyp{}occurrence statistic by
introducing a direct initialization method from the raw object-example data for
ENN. This allows us to avoid creating and storing $\BC$, which is a burden of
memory when $N$ becomes sufficiently large. In Algorithm \ref{alg:enn}, $\BC$
only appears in the initial truncated eigen\hyp{}decomposition, after which we
maintain the compressed operator $\BC_{op}$ independent of it. On the other
hand, we just need the matrix\hyp{}vector multiplication by $\BC$ for the
iterative methods in initialization. Using the generative formula in Equation (
\ref{eqn:unbiased}), we are able to implicitly apply $\BC$ to vectors as an
outer\hyp{}product plus diagonal operator, in terms of $\BH$, at $\calO(NMK)$
computation cost. To further reduce the number of times the operator is applied,
we adopt the one-pass randomized eigen\hyp{}decomposition by Halko et al. 
\cite{halko2011finding}. This technique enables us to initialize with a single
pass over the dataset, without concurrently storing the entire $\BH$ in memory.
A limitation is when the number of clusters is large and the gap between the $K$
\hyp{}th eigenvalue and the ones below is small, we will have to incorporate a
few power iterations for refinement, as suggested by the original paper. This
will result in a multi\hyp{}pass method, but still far more efficient on large
object size and parallelization-friendly.