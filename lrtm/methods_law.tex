The output for both methods in \cref{lrtmsec:lrr} is a compressed 
co\hyp{}occurrence matrix $C \Eq YY^T$. In this section, we present the 
\textbf{Low\hyp{}rank Anchor Word algorithm (LAW)} that reduces the time
complexity of finding anchor objects from $\calO(N^2K)$ to $\calO(NK^2)$ by
taking advantage of this form. We note that LAW applies whenever $C$ is in a
low\hyp{}rank representation, which does not have to be derived from our
methods. Moreover, it is exact for non\hyp{}negative $C$, but robust in 
practice when we allow small negative entries in $C$, as in the case with ENN.

\begin{figure}[htbp]
	\begin{algorithm}[H]
		\SetKwInput{KwInput}{Input}
		\SetKwInput{KwOutput}{Output}
		\DontPrintSemicolon
    	\KwInput{Object co-occurrence $C \Eq Y Y^T$}
		\KwOutput{Anchor objects $S \Eq \{s_1, ..., s_K\}$\\ 
		\hspace{36px} Latent clusters $B \in \BBR^{N \Times K}$\\
		\hspace{36px} Cluster correlations $A \in \BBR^{K \Times K}$}
		\Begin{
			Calculate row sums $d = Y(Y^Te)$.\;
			Normalize $\Ybar \leftarrow \diag(d)^{-1}Y$.\;
			\vspace{3px}
			Compute QR decomposition of $Y = QR$.\;
			Form $X = \Ybar R^T$.\;
			Select $S$ using column pivoted QR on $X^T$.\;
			\vspace{2px}
			Solve $n$ simplex\hyp{}constrained least square problems to minimize $
			\norm{X-\Bbreve X_{S\ast}}_F$.\;
			\vspace{2px}
			Recover $B$ from $\Bbreve$ using Bayes' rule.\;
			Recover $A = B_{S\ast}^{-1}Y_{S\ast}Y_{S\ast}^TB_{S\ast}^{-1}$.
		}
		\caption{Low-rank AW (LAW)}
	  \label{alg:law}      
	\end{algorithm}
	\vspace{2cm}
	\begin{algorithm}[H]
		\SetKwInput{KwInput}{Input}
		\SetKwInput{KwOutput}{Output}
		\DontPrintSemicolon
    \KwInput{Raw object-example $H \in \BBR^{N \Times M}$}
		\KwOutput{Anchor objects $S \Eq \{s_1, ..., s_K\}$\\ 
		\hspace{36px} Latent clusters $B \in \BBR^{N \Times K}$\\
		\hspace{36px} Cluster correlations $A \in \BBR^{K \Times K}$}
		\Begin{
		  Get $\Hhat, \Hhat_{\diag}$ from $H$ by \cref{eqn:unbiased}.\;
		  \vspace{1px}
			$C^{op}: x\rightarrow\Hhat (\Hhat^Tx) - \Hdiag x$\;
			\vspace{3px}
			$(U,\Lambda_K) \leftarrow$ Randomized\hyp{}Eig($C^{op}, K$)\;
			\vspace{3px}
			Initialize ENN with $U, \Lambda_K$.\;
			\vspace{3px}
			$Y\leftarrow$ ENN\hyp{}rectification\;
			\vspace{3px}
			$(S, B, A) \gets $LAW($Y)\;\;$ (\cref{alg:law})
		}
		\caption{Low-rank JSMF (LR-JSMF)}
	\label{alg:lr-jsmf}        
	\end{algorithm}
\end{figure}

The first step is to $L_1$\hyp{}normalize the rows of $C$. Given $C \geq 0$, 
the $L_1$\hyp{}norm of each row is simply the sum of all its entries, so we can
calculate the row norms by $d \Eq Y(Y^Te)$. To obtain the normalized $C$, we
simply scale the rows of $Y$, and $\Cbar \Eq (\diag(d)^{-1}Y)Y^T \Eq \Ybar 
Y^T$. These steps cost $\calO(NK)$.

Next, we need to apply column pivoted QR to \smash{$\BCbar^T$} in order to
identify the pivots as our anchor objects $S$. By taking the QR decomposition
$Y \Eq QR$, $\Cbar^T$ can be further transformed into $QR\Ybar^T$. Notice that
$\Cbar^T$ is an orthogonal embedding of $X^T = R\Ybar^T$ onto a
higher\hyp{}dimensional space, which preserves the column $L_2$\hyp{}norms.
\cref{lem:qrequiv} shows that column pivoted QR on $\Cbar^T$ and on $R\Ybar^T$
are equivalent, which allows us to lower the computation cost from $\calO(N^2K)$
to $\calO(NK^2)$.

\begin{lemma}\label{lem:qrequiv}
  Let $S$ be the set of pivots that have been selected by column pivoted QR on
  $\Cbar^T = QX^T$. Given the QR decomposition, $\Xbar_{S\ast}^T = PT$, then
  $\Cbar_{S\ast}^T = (QP)T$ is the corresponding QR decomposition for the
  columns of $\Cbar$. For any remaining row $i\in [N]\setminus S$,
  \begin{equation}
  	\norm{(I-PP^T)\Xbar_{i\ast}^T}_2 = \norm{\left(I-(QP)(QP)^T\right)\Cbar_
  	{i\ast}^T}_2\,.
  \end{equation}
  Therefore, the next column pivot is identical for $\Cbar^T$ and $\Xbar^T$. 
  By induction, column pivoted QR on $\Cbar^T$ and $\Xbar^T$ return the same
  pivots.
\end{lemma}
\begingroup
\allowdisplaybreaks
\begin{proof}
  Because both $Q$ and $P$ have orthonormal columns,
  \begin{equation}
    (QP)^T(QP) = P^T(Q^TQ)P = P^TP = I\,.
  \end{equation}
  Thus, $QP$ and $T$ forms the QR decomposition of $\Cbar_{S\ast}^T$. The
  residual of a remaining column $i\in[N]\setminus S$ is $\left(I-(QP)
  (QP)^T\right) \Cbar_{i\ast}^T$ and $(I-PP^T)\Xbar_{i\ast}^T$ for $\Cbar^T$ and
  $\BXbar^T$, respectively. Simplify the former gives us
  \begin{align*}
    &\left(I-(QP)(QP)^T\right)\Cbar_{i\ast}^T \\
    ={} & \left(I-(QP)(QP)^T\right)Q\Xbar_{i\ast}^T \\
    ={} & Q\Xbar_{i\ast}^T - QPP^TQ^TQ\Xbar_{i\ast}^T\\
    ={} & Q(I-PP^T)\Xbar_{i\ast}^T
  \end{align*}
  Finally,
  \begin{equation}\label{eqn:normpres}
    \norm{\left(I-(QP)(QP)^T\right)\Cbar_{i\ast}^T}_2^2 = \Xbar_{i\ast} 
    (I-PP^T)Q^TQ (I-PP^T)\Xbar_{i\ast}^T = \norm{(I-PP^T)\Xbar_{i\ast}^T}_2^2
  \end{equation}
  Because the next pivot is selected as the column whose residual has the
  largest $L_2$\hyp{}norm, Eq. \ref{eqn:normpres} indicates that the same pivot
  will be selected for $\Cbar^T$ and $\Xbar^T$. Inductively, the anchors $S$
  recovered by column pivoted QR on those matrices are equivalent.
\end{proof}
\endgroup

Following the recovery of $S$, AW solves $N$ independent 
simplex\hyp{}constrained least square problems $\norm{\Cbar_{i\ast} - \Bbreve_
{i\ast}^T\Cbar_{S\ast}}_2$. Again we can leverage the $L_2$\hyp{}norm preserving
property, 
\begin{equation}\label{eqn:lrls}
  \norm{\Cbar_{i\ast} - \Bbreve_{i\ast}^T\Cbar_{S\ast}}_2 = \norm{X_{i\ast}Q^T
  - \Bbreve_{i\ast}^TX_{S\ast}Q^T}_2 = \norm{X_{i\ast} - \Bbreve_{i\ast}^T X_
  {S\ast}}_2
\end{equation}
and reduce the dimension of the least\hyp{}square problems from $N$ to $K$,
thereby the complexity from $\calO(N^2K)$ to $\calO(NK^2)$. The remaining part
of the algorithm follows exactly as AW.

\paragraph{Low\hyp{}rank Joint Stochastic Matrix Factorization (LR\hyp{}JSMF)} 
We complete our scalable framework of processing co\hyp{}occurrence statistic by
introducing a direct initialization method from the raw object\hyp{}example data
for ENN. This allows us to avoid creating and storing $C$, which is a burden
of memory when $N$ becomes sufficiently large. In \cref{alg:enn}, $C$ only
appears in the initial truncated eigendecomposition, after which we maintain the
compressed operator $C^{op}$ independent of it. On the other hand, we just need
the matrix\hyp{}vector\hyp{}multiplication by $C$ for the iterative methods in
initialization. Using the generative formula in \cref{eqn:unbiased}, we are able
to implicitly apply $C$ to vectors as an outer\hyp{}product plus diagonal
operator, in terms of $H$, at $\calO(NMK)$ computation cost. To further reduce
the number of times the operator is applied, we adopt the one\hyp{}pass
randomized eigendecomposition by \citet{halko2011finding}. This technique
enables us to initialize with a single pass over the dataset, without
concurrently storing the entire $H$ in memory. A limitation is when the number
of clusters is large and the gap between the $K$\hyp{}th eigenvalue and the ones
below is small, we will have to incorporate a few power iterations for
refinement, as suggested by the original paper. This will result in a 
multi\hyp{}pass method, but still far more efficient on large object size and
parallelization\hyp{}friendly.